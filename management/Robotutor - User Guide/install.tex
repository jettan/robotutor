\section{Aquiring the Robotutor software}

\section{Preparing the robot}
Note: If a there is a NAO available which has the robotutor software already installed, preparing the NAO might not be necessary.

\subsection{Connecting to the robot}
To install the software on the robot we must first connect with it. Two types of connection are needed; file transfer and command transfer. There are serveral implementations available for this. For example, on Linux the scp and ssh commands can be used, on Windows WinSCP and PuTTY can be used. Using these methodes a connection should be made to the NAO. The IP address of the NAO can be found most easily by pressing it's chest button and listening to his response. Alternativly Choreographe or arpscan could be used. The default SSH user/password is nao/nao. 

\subsection{Installing software on the robot}

%schrijven naar var\www dus moet writeable zijn robotutor-server. 

Copy the robotutor tar file \emph{robotutor-setup.tar} to the NAO  (Using scp/WinSCP) and run the command "\emph{tar -xvzf robotutor-setup.tar}" (Using ssh/PuTTY)

\section{Preparing the local computer}

%Local engine
%Turning point - if nessesacry
%Pulse Audio