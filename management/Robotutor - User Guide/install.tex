
\section{Preparing the robot}
Note: If a there is a NAO available which has the robotutor software already installed, preparing the NAO might not be necessary.

\subsection{Connecting to the robot}
To install the software on the robot we must first connect with it. Two types of connection are needed; file transfer and command transfer. There are serveral implementations available for this. For example, on Linux the scp and ssh commands can be used, on Windows WinSCP and PuTTY can be used. Using these methodes a connection should be made to the NAO. The IP address of the NAO can be found most easily by pressing it's chest button and listening to his response. Alternativly Choreographe or arpscan could be used. The default SSH user/password is nao/nao. 

\subsection{Installing software on the robot}

Copy the robotutor tar file \emph{robotutor-setup.tar} to the NAO  (Using scp/WinSCP) and run the command "\emph{tar -xvzf robotutor-setup.tar}" (Using ssh/PuTTY)
Additionally, if you want to take pictures with the robot, these are stored in \textbackslash\textbackslash var\textbackslash\textbackslash www. This requires that this directory is writeable.
%This tar should contain the behaviors.

\section{Preparing the local computer}
If you want to playback the NAO's voice and sound via the local computer instead of the NAO's interal loudspeakers, you will need to install pulse audio on the local computer. The robotutor .tar file contains pulse audio for the NAO. You will need to connect the NAO to the computer's pulse audio server. As said earlier, the computer must run Powerpoint and, if used, Turingpoint 5.1. Note that Turningpoint must be configured to start with PowerPoint. The  RoboTutor software can be used without installation, as long as the files are available on the presentation computer.

%Local engine
%Turning point - if nessesacry
%Pulse Audio