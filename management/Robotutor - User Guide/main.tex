\documentclass[a4paper, 10pt, twoside]{report}
\usepackage{a4wide}             % voor wat beter gevulde a4-tjes
%\usepackage[dutch]{babel}       % voor Nederlandse woordafbreking
\usepackage{amsmath,xfrac}
\usepackage{SIunits}
\DeclareMathOperator{\sgn}{sign}
\usepackage[T1]{fontenc}

 \renewcommand{\familydefault}{\sfdefault}

\usepackage{etoolbox}
\makeatletter
\patchcmd{\chapter}{\if@openright\cleardoublepage\else\clearpage\fi}{}{}{}
\makeatother


\usepackage{cancel}             % voor het `wegstrepen' van tekst (bijv. in formules)
\usepackage{graphicx} 		    % voor afbeeldingen in pixelformaat (bmp, jpg enz.)
\usepackage{subfig}             % om plaatjes naast elkaar te kunnen zetten
\usepackage[utf8]{inputenc}
%\usepackage[small, bf, hang]{caption} % Voor netter onderschrift bij o.a. afbeeldignen.
\usepackage{fancyhdr}           % voor kop- en voettekst
\usepackage{rotating}           % voor het schuin zetten van teksten (bruikbaar bij bijvoorbeeld tabellen
\usepackage{wrapfig}
\usepackage{eso-pic}
\usepackage{colortbl}	        % Voor gekleurde tabellen							
\usepackage{color}		        % Voor gekleurde tekst
\usepackage{eurosym}
\usepackage{latexsym}
\usepackage[pagebackref=true]{hyperref}
\usepackage{verbatim}	        % Voor comment env!

% Shit voor Listings
\usepackage{listings}
\definecolor{dkgreen}{rgb}{0, 0.6, 0}
\definecolor{blue}{rgb}{0, 0, 1}

\lstset{
language=Matlab,
basicstyle=\small\ttfamily,
numbers=left,                       
numberstyle=\small\ttfamily,                            
showspaces=false,                  
showstringspaces=false,             
showtabs=false,                     
tabsize=2,                          
captionpos=b,                      
breaklines=true,                    
breakatwhitespace=false,            
title=\lstname,                    
keywordstyle=\color{blue},
commentstyle=\color{dkgreen},
flexiblecolumns=true,
}

% Configuratie voor TiKZ en PGFPlots
\usepackage{tikz}
\usetikzlibrary{arrows,automata, shapes}
\tikzstyle{vertex}=[circle]
\usepackage{pgfplots}


% Headers instellen.
\pagestyle{fancy} 				        % Voor de opmaak van kop- en voetteksten.
\setlength{\headheight}{0pt}		% Zoals: paginanummers, hoofdstukken enz.  
\fancyhf{}
\fancyfoot[C,R,L]{}				        % De plekken waar geen tekst moet staan eerst leegmaken
\fancyhead[OR]{\slshape\rightmark}		% \rightmark = Paragraaftitel
\fancyhead[EL]{\slshape\leftmark}		% \leftmark = Hoofdstuktitel
\fancyfoot[C]{\thepage}				    % \thepage = paginanummering
\renewcommand{\headrulewidth}{0,4pt}
\renewcommand{\thetag}{\nonumber}

\setlength{\parindent}{0pt}             %geen identatie bij nieuwe paragrafen

\usepackage{pifont} 
\newcommand{\tick}{\ding{52}}

\hypersetup{pdftitle=Robotutor - User guide, pdfauthor=Robotutor Group, colorlinks=true}
\graphicspath{{images/}}

\makeindex

\newcommand{\us}{\,$\mu$s\xspace}
%\usepackage{cmbright} %andere font
\newcommand\BackgroundPic{\put(-5,0){\parbox[b][\paperheight]{\paperwidth}{\vfill \raggedright \includegraphics[width=1.05\paperwidth,height=\paperheight]{images/front.pdf}}}}
	
\usepackage{etoolbox}
\makeatletter
\preto{\@verbatim}{\topsep=0pt \partopsep=0pt }
\makeatother

\begin{document}

\begin{titlepage}

\AddToShipoutPicture*{\BackgroundPic}
{\color{white}\ \\[.5cm]
\begin{Huge}Robotutor\\[.8cm] \end{Huge}
\begin{huge}User Guide\end{huge}\\[1.6cm]
\begin{large}\color{cyan}\today \end{large}}\\[3cm]
\begin{tabbing}
\hspace*{5cm}\= \kill

Anass Drif\\
Hans Gaiser\\
Jethro Tan\\
Pim Veldhuisen\\ 
Koen V. Hindriks \\

\end{tabbing}

\vspace{3 cm}

\end{titlepage}
%\input{versieoverzicht}
\setcounter{tocdepth}{3}
\tableofcontents

\chapter*{Introduction} 
This document describes the actions required to use the robotutor presentation software. It is intended for end-users and hence does not go into detail on technical aspects.\\
\chapter{Required Hardware}
Robotutor requires an Aldebaran NAO robot with an Intel CPU running software version 1.14. Other software versions might also work but are untested. A NAO with motorized fingers is recommended for added motion interaction.\\
\\
To control the robot and run the presentation, a local computer is also needed. This can be any device running a recent version of Windows. It must be capable of audio playback and have Microsoft Powerpoint installed. Other presentation software will not work. If the Turning Point audience response system will be used in the presentation, version 5.1 of its software must also be installed.

%NAO
%1.14
%Fingers

\chapter{Installing Robotutor}

\section{Preparing the robot}
Note: If a there is a NAO available which has the robotutor software already installed, preparing the NAO might not be necessary.

\subsection{Connecting to the robot}
To install the software on the robot we must first connect with it. Two types of connection are needed; file transfer and command transfer. There are serveral implementations available for this. For example, on Linux the scp and ssh commands can be used, on Windows WinSCP and PuTTY can be used. Using these methodes a connection should be made to the NAO. The IP address of the NAO can be found most easily by pressing it's chest button and listening to his response. Alternativly Choreographe or arpscan could be used. The default SSH user/password is nao/nao. 

\subsection{Installing software on the robot}

Copy the robotutor tar file \emph{robotutor-setup.tar} to the NAO  (Using scp/WinSCP) and run the command "\emph{tar -xvzf robotutor-setup.tar}" (Using ssh/PuTTY)
Additionally, if you want to take pictures with the robot, these are stored in \textbackslash\textbackslash var\textbackslash\textbackslash www. This requires that this directory is writeable.
%This tar should contain the behaviors.

\section{Preparing the local computer}
If you want to playback the NAO's voice and sound via the local computer instead of the NAO's interal loudspeakers, you will need to install pulse audio on the local computer. The robotutor .tar file contains pulse audio for the NAO. You will need to connect the NAO to the computer's pulse audio server. As said earlier, the computer must run Powerpoint and, if used, Turingpoint 5.1. Note that Turningpoint must be configured to start with PowerPoint. The  RoboTutor software can be used without installation, as long as the files are available on the presentation computer.

%Local engine
%Turning point - if nessesacry
%Pulse Audio
\chapter{Preparing A Presentation}
The first step in creating a robotutor presentation is creating a regular slideshow in Microsoft PowerPoint. It can also include polling sildes made by the Turning Point software. The robot will later use this slideshow in his presentation. The next step is to create an presentation script. This is a text file in wich the robots behavior is described. Most importantly it contains the text that the robot will speak. Within this text commands can be embedded.


\section{Commands}
\subsection{slide}
The command "slide" can be used to control the slideshow. This command has 3 modes; it can be used without argument to go to the next slide, it can be used with a number to go to a specific slide, or it can be used with an +/- sign followed by a number to go to a slide relative to the current slide. See the following examples.


\begin{table}[h!]
\begin{tabular}{|l|l|}
\hline
Command & Result\\ \hline
\{slide\}& Go to the next slide\\ 
\{slide|1\}  & Go to slide 1\\ 
\{slide|7\}  & Go to slide 7\\ 
\{slide|+1\} & Skip 1 slide ahead/Go to the next slide \\ 
\{slide|+8\} & Skip 8 slides ahead\\ 
\{slide|-5\} & Skip 5 slides backward\\
\hline
\end{tabular}
\end{table}


\subsection{Behaviors}
There are three types of behaviors that the Robotutor uses. The first type consists of special behaviours that are not meant to be used during the presentation, but to start or stop the robot.
The second type consists of specific behaviours. These are behaviours that contain specific gestures that may be used in the presentation, such as point to the left or the right, the NAO introducing itself, and the NAO indicating the position of it's sensors. In the script, these behaviours can be exectuted as in the following example: "{behavior|robotutor/specific/Me}".
The third and last type of behaviours are the generic ones. The are small body movements that can be exectuted randomly throughout the presentation to make the robot look more lively.


\subsection{Play a sound}
The Robotutor can play wav files from it's internal memory using the command

\subsection{Do a Quiz}

\subsection{Take an Image}


%Make a presentation in powerpoint
%Make a script
% Commands
\chapter{Running A Presentation}
%\section{Running the Robotutor software on the robot}
%Connect to the robot using PuTTY /ssh
%ssh to nao
%run the client

%\section{Running the Robotutor software on the local computer}
On the local computer, run the provided software. This provides you with a graphical interface to run the presentation. In the interface, simply enter the NAO's IP-adress, select the presentation script and start. the presentation can be interrupted at will.

%run the gui
%interrupt


%\nocite{*}
%\bibliographystyle{IEEEtran}
%\bibliography{Bibliography}
%\addcontentsline{toc}{chapter}{Bibliography}

%\appendix
%\input{appendix.tex}

\end{document}