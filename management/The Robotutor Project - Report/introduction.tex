%What comes to mind when you think about education? Chances are that it involves an experienced teacher talking to a number of students. But why is this the most common way of teaching? In most cases all of the concepts the teacher is conveying are also available in books, explaining them clearly, concisely and at the student's own pace. Still, most students choose to listen to a teacher instead of reading a book, even when attendance is optional. Why? Because teaching is so much more than saying words. Teaching is about interaction, about asking questions, about humor; about connecting with the audience.

The world of education is changing rapidly through new technologies. Massive open on-line courses (MOOCs) have become very popular with renowned institutions successfully launching such courses with thousands of students participating simultaneously. These courses are a great way to reach large audiences but allow only for a limited amount of interaction. More specifically, they lack the embodied and physical presence of a teacher that has been reported to make a difference \cite{Hoffmann2011}. Here clearly is an opportunity for the use of robots that can provide such \textit{embodied interaction}. Robots have been used as teachers before \cite{Castellano2013,Han2005,Saerbeck2010} but the focus in this work has been on one-to-one interaction whereas we focus on one-to-many interaction. The work reported in \cite{Castellano2013}, for example, lists several requirements in which empathy, detection of learner affective state, and social bonding are emphasized. The challenges of \textit{interacting with an audience} are different and require different social behaviours, a focus on maintaining audience attention and on ways to engage and entertain the audience.

One of the drivers that motivated the RoboTutor project has been the observation that in a lecture about robotics it is natural to show a robot and demonstrate some of its capabilities. The use of the humanoid robot Nao inspired us to extend the role of the robot in the lecture even further and the idea was born to have the robot \textit{present itself} in a classroom with students that take the course. In this setting, it is important that the robot not only presents a monologue but also interacts with its audience and with the teacher that it assists in giving the lecture. The primary goal thus has been to develop a robot that can give an interactive presentation together with a human lecturer. In principle, however, the system we have built can also be applied in other infotainment settings in which a robot interacts with an audience. Other goals of the project are to study human-robot audience interaction \cite{Knight2011,Mutlu2006} and to develop a framework for effective audience interaction similar to the behaviour toolkit discussed in \cite{Huang2012}. In line with our current focus, this mainly involves identifying and integrating the social behaviours that are required for educational and infotainment contexts.

%The approach taken in the project begins with considering the capabilities and features that a robot needs to be able to give a presentation that is perceived as interesting. Based on these requirements, software is designed and developed for the Nao. The realised system is evaluated by real audiences throughout the process, and based on feedback from these sessions the software is continuously improved.

In Section \ref{sec:architecture} we discuss the system architecture of the robotic tutor. Section \ref{sec:evaluation} presents some preliminary findings from several lectures given by our RoboTutor. In Section \ref{sec:conclusion} we conclude and briefly discuss how we want to extend the system in the near future.

