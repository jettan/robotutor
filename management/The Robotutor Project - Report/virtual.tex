The deployment of virtual teacher could provide many advantages over their human counterparts. In line with other applications for virtual agents, greater cost efficiency and availability can be used to either decrease expenditures or inprove utility. Focussing or the later, virtual teachers could more easily give a certain lecture at multiple timeslots or repeat and elaborate lectures for students having trouble understanding certain concepts.\\

Another great opportunity lies in the optimization of education. Virtual teachers could deliver the same lessons consistenly in great quantaties. The effects of minor changes in these lessons can thus be measured more reliably than in human teachers, where there is an natural variation in every instance that makes it hard to determine the effect of a change without taking in to account other differences. Expanding on this concept, machine learning for robotic teachers might even be envisaged. When the virtual agents recieve feedback on there performace, such as audience scores or even test results, the teachers could attempt to improve themselves in teaching there subject material. If the gathered information is shared across many teachers, this could lead to a self optimizing system of education. 