One finding during early trials of the system was that the robot needs to be continuously moving because otherwise attention of the audience drops. This poses several technical challenges again. First, the motors of the joints can overheat. In our case, we can actually use such motor ``failure" to explain something about the workings of the robot joints. Second, we found that the robot cannot simply perform random gestures but the timing and type of gestures displayed should make sense and match the spoken text.

The RoboTutor has been put to the test in front of a real audience a number of times throughout its development. The duration of the lecture presentations varied from about five to twenty minutes. In general we have been surprised by the robot's ability to capture audience attention and to convey information. This was true even though we used the out-of-the-box synthesized voice of the Nao robot with almost no modifications; participants in our initial try-outs agreed that the robot's voice does not pose a big problem. 

During the demonstrations the quizzes turned out to be a great source of interaction. The questions asked by the Nao trigger the audience to think about what the robot said and to use the information provided in the lecture. But more importantly the response of the RoboTutor to the answers received from the audience shows that the RoboTutor is able to interact with its audience, and, as a result, the audience feels engaged. In addition it provides a tool to evaluate and analyse knowledge transfer from the RoboTutor to its audience and to study how to effective educate an audience.

It appears also quite important that the robot somehow conveys that it is aware of its audience. Simple mechanisms could enhance the feeling of robot awareness in its audience. For example, the robot could look at and point more explicitly to the slides during the presentation and then look back at audience. This type of behaviour would help suggest that the robot is aware of where it is looking at. It also remains a challenge to make the robot appear to look more or less randomly at participants in its audience and thus improve the feeling that the robot is aware of its audience. The robot could also be made to appear more situation-aware by including a simple event-based mechanism that allows the robot to await for an event and to respond to an event.
