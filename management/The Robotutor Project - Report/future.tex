The RoboTutor is promising but a lot of work remains to be done to develop more engaging and interactive robotic tutors. An important aspect of this development would be the realisation of a system for beter sensing of the audience. One prospective way of doing this would be the use of an advanced vision system (possibly in combination with an external camera). Such a system would allow the RoboTutor to increase its spatial awareness, by enabling it to locate the teacher and individuals in its audience. Moreover, the robot could monitor the audience and respond to it. For example, by seeing a raised hand the RoboTutor might detect that a student in the classroom wants to ask a question. Ofcourse it is then imperative the robot can also respond to this event and allow the student to speak. This would mean the script that the robot follows needs to be interrupted at certain times. To facilitate this, a control layer would be needed to coordinate the alternation between the script and audience events. Last but not least, after allowing the student to speak, the RoboTutor should also be able to respond to the inquiry. This would require integrating advanced state-of-the-art question answering systems.