When opting for the use of modern technologies in education the choice for embodied robotics is not immediately obvious. Teaching tools such as videos, presentations, readings and quizzes can all be delivered through the familiar and readily available channel of the internet onto computers. Even when choosing for a solution that involves some form artificial intelligence, an abstract delivery system seems the most obvious choice. Although an experience involving embodied robotics is often more costly both in terms of money and effort, the greater cost can be amply offset by the benefits of embodiment.\\

Research show that embodied characters can increase performance of interacting subjects. \cite{Bartneck2002} This is most likely due to an increased level of social presence, motivating the people involved to put in more effort. Other studies show that embodied interaction is also experienced as more enjoyable, both in entertainment \cite{Pereira2008} and cognitive situations.\cite{Tapus2009}

These advantages are essential in educational situations and thus make the use of embodied agents preferable to virtual agents.

