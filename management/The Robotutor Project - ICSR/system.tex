\subsection{Architecture}
The Robotutors presentations are based on scripts. These are human made files containting the text of the presentation. This text is synthesised to speech by the NAO's engine during the presenation. Additonally the text also contains a number of commands. These are actions the robot will perform during the lecture, such as its movements. The pacing of the slideshow is also controlled using these commands.\\
During the presentation, two pieces of software facilitate the execution of the script. One part runs on the NAO itself, another on the computer that is used for the presentation. The computer is connected to the beamer or screen and runs the slideshow.\\

The application that is executed on the computer that runs the slideshow is created using the Qt-toolkit, which is a graphical toolkit written in C++ and is commonly used for cross platform applications. Through this application the user can control the Robotutor application running on the robot. These controls include connecting to the robot, running a script, pausing a script and stopping a script. When this application is launched, it will automatically also launch a Powerpoint instance with which it maintains a communication link. This allows the application to control the Powerpoint application, by for example triggering a next slide event or even adding slides. When the user closes the application, the Powerpoing application is also automatically closed. In addition, this application contains a simple text editor which the user can use to create a script. This text editor has some simple highlighting to indicate which parts are text that the NAO will speak out, which parts are behaviours the NAO will perform and which parts have been commented out and will be ignorred.

When the user decides it is time to start the script, the script is sent to the robot. The application running on the robot receives this messages in plain text and interprets the message to determine what to do. This will mean certain blocks of text are send to the speech engine running on the NAO and the behaviours are stored in a queue and will be executed when the speech engine notifies it is time for that behaviour to be executed.

The two applications talk to eachother through a protocol called Google Protocol Buffers. Protocol buffers is a method to serialize structured data to allow for easy sending over a network channel. In this case it is used to send a script or commands from one application to the other.

% The two pieces of software communicate using protobuf, this solution was being chosen because..

\subsection{Features}
\subsubsection{Powerpoint}
The robot can go to the next slide, step back or forward a specific number of slide, or go to a certain slide. Additonally, slide with new content can be dynamically created.

\subsubsection{Noise level}
The system can monitor the noise level in the room, and respond accordingly if there is to much noise. Additonally the noise level can be visualized live on a Powerpoint slide.

\subsubsection{Turningpoint}
The Turningpoint system consists of a number of response cards, small electronic devices, connected to a computer. The response card are handed out to people in the audience, who can use them to digitally respond to a question. Turningpoint then aggregates and presents the results. This can be used by a lecturer to check the audiences opinon or knowledge. The use of this technology is integrated into the Robotutor system. Using it, the NAO can ask questions and interpret the results. This provides a great method for interaction with the audience.

\subsubsection{Pulse Audio}
In large rooms or lecture halls, the NAO's internal speakers might not have enough volume to server the audience properly. For these kind of situations audio streaming has been implemented. Using pulse audio, the NAO's sound can be redirected to a different set of speakers.
