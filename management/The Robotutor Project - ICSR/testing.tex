The Robotutor has been put to the test in front of a real audience a number of times throughout its development. In general people are pleasantly surprised over the robots ability to capture the audience's attention and to convey information. The syntesized voice of the robot was beforehand considered as a possible source of irritation, but the respondents widely agreed that this was not a problem. During the tests the importance of the robot's movement turned out to be even greater than anticipated. During any presentation body language is terribly important, and for the robot this is no different. However it remains difficult to make the NAO appear natural through subtle movements, effort has been made to ensure that the robot doesn't appear frozen and supports its text through gestures. \\

During the evaluations the Turningpoint quizzes turned out to be a great source of interaction. The questions asked by the NAO  not only triggered the audience to think about what the robot said and to recapitulate the information in the lecture, but also showed in a very direct way that the Robotutor can receive input from the audience and respond accordingly. This helped the audience to feel connected to the robot and contributed greatly to the level of interaction in the lecture.