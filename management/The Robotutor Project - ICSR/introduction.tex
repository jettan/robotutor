What comes to mind when you think about education? Chances are that it involves an experienced teacher talking to a number of students. But why is this the most common way of teaching? In most cases all of the concepts the teacher is conveying are also available in books, explaining them cleary, concisely and at the student's own pace. Still, most students choose to listen to a teacher instead of reading a book, even when attendance is optional. Why? Beacuse teaching is so much more than saying words. Teaching is about interaction, about asking questions, about humor; about connecting with the audience.\\

The world of education is changing through new technologies. Massive open online courses are very popular these days, with renowned institutions joining the pioneers. These courses are a great way to reach large audiences, but often lack interaction. Here lies an opportunity for robots, who could provide embodied interaction with an audience. However, interaction is not easy to achieve for robots. A lot of research exist on interaction between robots and single humans, but research on interaction between robots and audiences is scarce. The NAO robot has mainly been used for one-on-one interaction with human beings. In entertainment however, the main goal is to have one-to-many interaction, similar to the interaction between teacher and students. The goal of the Robotutor project is to develop a robot lecturer, that can assist a professor with his lectures. The benefits of this project are twofold: Sophisticated methods for one-to-many interaction are developed and the project itself is a very insightful process in which it is necessary to reflect on the question what really is important in education.

