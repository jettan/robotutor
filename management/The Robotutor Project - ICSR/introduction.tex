What comes to mind when you think about education? Chances are that it involves an experienced teacher talking to a number of students. But why is this the most common way of teaching? In most cases all of the concepts the teacher is conveying are also available in books, explaining them clearly, concisely and at the student's own pace. Still, most students choose to listen to a teacher instead of reading a book, even when attendance is optional. Why? Because teaching is so much more than saying words. Teaching is about interaction, about asking questions, about humor; about connecting with the audience.\\

The world of education is changing through new technologies. Massive open online courses are very popular these days, with renowned institutions joining the pioneers. These courses are a great way to reach large audiences, but often lack interaction. Here lies an opportunity for robots, who could provide embodied interaction with an audience. However, interaction is not easy to achieve for robots. A lot of research exist on interaction between robots and single humans, but research on interaction between robots and audiences is scarce. This is also true for the NAO; when it is used for interaction with human beings it's usually in the form of one-on-one interaction. In education however, interaction exist often in the form of one-to-many interaction, such as between a teacher and a group of students. The Robotutor project explores the issue of robot-classroom interaction in a practical way. Its direct goal is to develop a robot that can give an interactive presentation to assist a professor with his lectures. The academic benefits of this project are twofold: research is done in the field of one-to-many interaction and the project itself is a very insightful process in which it is necessary to reflect on the question what really is important in education.\\

The approach taken in the project begins with considering the capabilities and features that a robot needs to give an interesting presentation. Based on these requirements software is designed and developed for the NAO. The realised system is evaluated by real audiences throughout the process, and based on feedback from these session the software is continuously improved.

