The Robotutor project explores the issue of robot-classroom interaction in a practical way. It is based on Aldebaran's NAO, a humanoid robot. The main goal of the project is to develop prototype software that enables a robot to give an engaging presentation to an audience. This software uses the robot's capabilities to interact with the audience and deliver content that has been developed beforehand. The content of the presentations is contained in a text-based script, wich is proccesed during the presentation. This enables easy creation of content without the need of much technical understanding of the NAO.\\
The goal of the project is mainly to explore future possiblities for the use of robots in education. The current potential for robots has been gauged and some of the challenges for the future have been identified. An additional goal would be to bring the software to a level where it can actually be used for complete lectures.\\
