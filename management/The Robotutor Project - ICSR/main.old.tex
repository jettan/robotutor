\documentclass[a4paper, 10pt, twoside]{report}
\usepackage{a4wide}             % voor wat beter gevulde a4-tjes
%\usepackage[dutch]{babel}       % voor Nederlandse woordafbreking
\usepackage{amsmath,xfrac}
\usepackage{SIunits}
\DeclareMathOperator{\sgn}{sign}
\usepackage[T1]{fontenc}


\usepackage{etoolbox}% http://ctan.org/pkg/etoolbox
\makeatletter
\patchcmd{\@makechapterhead}{\vspace*{50\p@}}{\vspace*{5pt}}{}{}%
\makeatother



\setlength{\parskip}{2pt}

\makeatletter
\renewcommand\chapter{\thispagestyle{plain}%
\global\@topnum\z@
\@afterindentfalse
\secdef\@chapter\@schapter}
\makeatother 


\usepackage[compact]{titlesec}  
\titlespacing{\section}{0pt}{0pt}{0pt}

 %\renewcommand{\familydefault}{\sfdefault}

\usepackage{cancel}             % voor het `wegstrepen' van tekst (bijv. in formules)
\usepackage{graphicx} 		    % voor afbeeldingen in pixelformaat (bmp, jpg enz.)
\usepackage{subfig}             % om plaatjes naast elkaar te kunnen zetten
\usepackage[utf8]{inputenc}
%\usepackage[small, bf, hang]{caption} % Voor netter onderschrift bij o.a. afbeeldignen.
\usepackage{fancyhdr}           % voor kop- en voettekst
\usepackage{rotating}           % voor het schuin zetten van teksten (bruikbaar bij bijvoorbeeld tabellen
\usepackage{wrapfig}
\usepackage{eso-pic}
\usepackage{colortbl}	        % Voor gekleurde tabellen							
\usepackage{color}		        % Voor gekleurde tekst
\usepackage{eurosym}
\usepackage{latexsym}
\usepackage[pagebackref=true]{hyperref}
\usepackage{verbatim}	        % Voor comment env!

% Shit voor Listings
\usepackage{listings}
\definecolor{dkgreen}{rgb}{0, 0.6, 0}
\definecolor{blue}{rgb}{0, 0, 1}

\lstset{
language=Matlab,
basicstyle=\small\ttfamily,
numbers=left,                       
numberstyle=\small\ttfamily,                            
showspaces=false,                  
showstringspaces=false,             
showtabs=false,                     
tabsize=2,                          
captionpos=b,                      
breaklines=true,                    
breakatwhitespace=false,            
title=\lstname,                    
keywordstyle=\color{blue},
commentstyle=\color{dkgreen},
flexiblecolumns=true,
}

% Configuratie voor TiKZ en PGFPlots
\usepackage{tikz}
\usetikzlibrary{arrows,automata, shapes}
\tikzstyle{vertex}=[circle]
\usepackage{pgfplots}


% Headers instellen.
\pagestyle{fancy} 				        % Voor de opmaak van kop- en voetteksten.
\setlength{\headheight}{0pt}		% Zoals: paginanummers, hoofdstukken enz.  
\fancyhf{}
\fancyfoot[C,R,L]{}				        % De plekken waar geen tekst moet staan eerst leegmaken
\fancyhead[OR]{\slshape\rightmark}		% \rightmark = Paragraaftitel
\fancyhead[EL]{\slshape\leftmark}		% \leftmark = Hoofdstuktitel
\fancyfoot[C]{\thepage}				    % \thepage = paginanummering
\renewcommand{\headrulewidth}{0,4pt}
\renewcommand{\thetag}{\nonumber}

\setlength{\parindent}{0pt}             %geen identatie bij nieuwe paragrafen

\usepackage{pifont} 
\newcommand{\tick}{\ding{52}}

\hypersetup{pdftitle=The Robotutor Project, pdfauthor=Robotutor Group, colorlinks=true}
\graphicspath{{images/}}

\makeindex

\newcommand{\us}{\,$\mu$s\xspace}
%\usepackage{cmbright} %andere font
\newcommand\BackgroundPic{\put(-5,0){\parbox[b][\paperheight]{\paperwidth}{\vfill \raggedright \includegraphics[width=1.05\paperwidth,height=\paperheight]{images/front.pdf}}}}

\hypersetup{
  colorlinks,
  linkcolor=black
}

\begin{document}

\begin{titlepage}

\AddToShipoutPicture*{\BackgroundPic}
{\color{white}\ \\[.5cm]
\begin{Huge}The Robotutor Project\\[.8cm] \end{Huge}
\begin{large}\color{cyan}\today \end{large}}\\[3cm]
\begin{tabbing}
\hspace*{5cm}\= \kill

Anass Drif\\
Hans Gaiser\\
Jethro Tan\\
Maarten de Vries\\
Pim Veldhuisen\\ 
\\
Under supervision of Dr. Koen Hindriks

\end{tabbing}

\vspace{10 cm}

\textbf{Abstract\\}
The Robotutor project consists of software developed for the NAO robot that enables it to give an educational presentation. The presentation consists mainly of the NAO talking and controlling an accompanying slideshow. During the presentation the NAO interacts with its audience in various ways to keep them engaged and entertained.

\end{titlepage}
%\input{versieoverzicht}
\setcounter{tocdepth}{3}
\tableofcontents

\chapter*{Introduction} 
\addcontentsline{toc}{chapter}{Introduction}

What comes to mind when you think about education? Chances are that it involves an experienced teacher talking to a number of students. But why is this the most common way of teaching? In most cases all of the concepts the teacher is conveying are also available in books, explaining them cleary, concisely and at the student's own pace. Still, most students choose to listen to a teacher instead of reading a book, even when attendance is optional. Why? Beacuse teaching is so much more than saying words. Teaching is about interaction, about asking questions, about humor; about connecting with the audience.\\

The world of education is changing through new technologies. Massive open online courses are very popular these days, with renowned institutions joining the pioneers. These courses are a great way to reach large audiences, but often lack interaction. Here lies an opportunity for robots, who could provide embodied interaction with an audience. However, interaction is not easy to achieve for robots. A lot of research exist on interaction between robots and single humans, but research on interaction between robots and audiences is scarce. The NAO robot has mainly been used for one-on-one interaction with human beings. In entertainment however, the main goal is to have one-to-many interaction, similar to the interaction between teacher and students. The practical goal of the Robotutor project is to develop a robot lecturer, that can assist a professor with his lectures. The academic benefits of this project are twofold: research is done in the field of one-to-many interaction and the project itself is a very insightful process in which it is necessary to reflect on the question what really is important in education.



\chapter*{System}
\addcontentsline{toc}{chapter}{System}

The RoboTutor system architecture consists of several components. The overall design has been based mainly on some practical considerations related to being able to present and manipulate a slideshow and to perform associated interactive behaviours on the robot itself.  The main components for handling a slideshow (we have used PowerPoint \cite{PowerPoint} and TurningPoint \cite{TurningPoint}) are run on a laptop computer whereas other interaction behaviours are run on the Nao itself. Fig. \ref{fig:system_overview} provides an overview of these components; in the figure, red arrows represent control flow and black arrows represent data flow. The robot and laptop computer communicate with each other through a protocol called Google Protocol Buffers \cite{protobuf}. Google Protocol Buffers provide a language- and platform-neutral tool that is easily extended and used by the RoboTutor system to exchange a script or commands between the laptop computer and the robot over a network channel.

\begin{figure}
\centering
	\includegraphics[scale=0.07]{images/system_overview.png}
	\caption{Overview of the RoboTutor System}\label{fig:system_overview}
\end{figure}

%
\paragraph{Script Engine and Editor}
The RoboTutor uses a script engine that runs on the Nao to present a lecture. A script consists of a series of commands, including the possibility to branch depending on input received during the presentation, changing slides in a slideshow and performing arbitrary behaviours that the Nao is capable of. An important part of a script is to control the ``body language" of the robot (e.g., gestures) to support the text spoken by the Nao. Currently, supported behaviours include various behaviours for moving the arm, legs and head together or separately as a (human) presenter would (e.g., for performing a pointing behaviour). Additional and more complex behaviours such as dance behaviours are easily integrated by means of Choregraphe. %\cite{choregraphe}.
 Other commands that are available include taking a picture of the audience, controlling the pace of the slideshow and pausing/resuming the execution of the script. Moreover, at any moment a human (teacher) can pause and resume the execution of the script using the script GUI or by touching the head of the Nao.

A simple GUI that can be launched on the laptop computer contains a text editor that can be used to create a script. %This editor has basic syntax highlighting capabilities (to differentiate, e.g., spoken text from behaviour commands and comments).
When the script is run, it is sent to the script engine on the robot. A custom script parser has been written in C++ which executes the script in real time. Plain text in the script is sent to the speech engine running on the Nao and the commands are being handled separately. In case a command to run a behaviour is received, it is added to a queue and is executed when the speech engine notifies it is time for that behaviour to be executed. Currently, synchronising text and behaviours of the Nao is done by letting the Nao perform a behaviour that matches the context of the text and the duration of the spoken text. Commands to change a slide are sent back to the laptop computer which forwards these to PowerPoint.

%
\paragraph{Presentation Application}
The laptop computer is connected to a beamer and is used to project the slideshow on a screen. The main component executed on the laptop used for presenting the slideshow is an engine that is created using the Qt-toolkit \cite{QT}. Qt is a graphical toolkit written in C++ that is commonly used for cross platform applications and supports the development of user interfaces. By means of the Qt user interface the user has control over the RoboTutor application code that runs on the robot itself. It facilitates connecting to the robot, and allows controlling the script by either pausing, running, or terminating a script. By launching the interface, PowerPoint is also automatically launched and a communication link between the engine and PowerPoint is created. This allows the laptop application to control the PowerPoint slideshow and trigger, e.g., a next slide event for moving to the next slide. The RoboTutor can move to the next slide, step back or forward an arbitrary number of slides, or go to a specific slide. It is even possible to dynamically create and add a slide to the slideshow while the robot is giving a presentation. One use case for this feature is to take a picture using one of Nao's cameras and display it on a newly created slide. This usually means the audience can see itself, creating a humorous moment; it, moreover, also to some extend suggests that the robot is aware of its audience.

%
\paragraph{Support for Quizzes}
The TurningPoint system consists of a number of response cards (small electronic devices) connected to a computer. These response cards are handed out to participants in the audience. The cards can be used to (digitally) indicate a choice in reply to a multiple-choice question that is projected on a slide. The results are received and aggregated by the TurningPoint. The TurningPoint software is an integral part of the RoboTutor system and enables the RoboTutor to ask questions and interpret the results. This in turn enables the RoboTutor to react and respond to audience input. The RoboTutor can also use this tool to decide on how to proceed in a lecture. For example, the robot can ask whether a concept is clear to the audience or more explanation is needed. Based on feedback from the audience, the robot can make a decision on how to proceed.

TurningPoint questions are started by a command in the script. The script engine running on the robot then requests the local engine to initiate a TurningPoint session through Powerpoint. After TurningPoint has polled the answers from the audience, the local engine component receives and processes an XML-file that contains the results of the quiz. These are then sent to the robot's script engine, which upon receiving the results makes a decision on what to do next.

%
\paragraph{Audio} The laptop computer can also be connected to an audio system if that is available in a classroom to route and enhance audio output of the Nao. In large rooms or lecture halls, the maximum volume of Nao's internal speakers may not be sufficient to enable the audience to follow the lecture properly. To handle these situations audio streaming has been enabled in the Nao. Using pulse audio, the Nao's sound can be redirected to a different set of speakers. 
The RoboTutor is also able to monitor the noise level in the room, and respond accordingly if there is too much noise. Additionally the noise level can be visualized live on a PowerPoint slide.


\chapter*{Testing and Evaluation}
\addcontentsline{toc}{chapter}{Testing and Evaluation}
One finding during early trials of the system was that the robot needs to be continuously moving because otherwise attention of the audience drops. This poses several technical challenges again. First, the motors of the joints can overheat. In our case, we can actually use such motor ``failure" to explain something about the workings of the robot joints. Second, we found that the robot cannot simply perform random gestures but the timing and type of gestures displayed should make sense and match the spoken text.

The RoboTutor has been put to the test in front of a real audience a number of times throughout its development. The duration of the lecture presentations varied from about five to twenty minutes. In general we have been surprised by the robot's ability to capture audience attention and to convey information. This was true even though we used the out-of-the-box synthesized voice of the Nao robot with almost no modifications; participants in our initial try-outs agreed that the robot's voice does not pose a big problem. 

During the demonstrations the quizzes turned out to be a great source of interaction. The questions asked by the Nao trigger the audience to think about what the robot said and to use the information provided in the lecture. But more importantly the response of the RoboTutor to the answers received from the audience shows that the RoboTutor is able to interact with its audience, and, as a result, the audience feels engaged. In addition it provides a tool to evaluate and analyse knowledge transfer from the RoboTutor to its audience and to study how to effective educate an audience.

It appears also quite important that the robot somehow conveys that it is aware of its audience. Simple mechanisms could enhance the feeling of robot awareness in its audience. For example, the robot could look at and point more explicitly to the slides during the presentation and then look back at audience. This type of behaviour would help suggest that the robot is aware of where it is looking at. It also remains a challenge to make the robot appear to look more or less randomly at participants in its audience and thus improve the feeling that the robot is aware of its audience. The robot could also be made to appear more situation-aware by including a simple event-based mechanism that allows the robot to await for an event and to respond to an event.


\chapter*{Conclusion}
\addcontentsline{toc}{chapter}{Conclusion}
The Robotutor software enables the NAO to interact with an audience in an engaging way. Using its flexibility it can be used in many different senarios, ranging from educational lectures to stand-up commedy. In all these roles it manages to capitvate its audience and inspire enthusiasm in people over the capabilities of robots.\\

There is still a lot of ground to cover for robotic teachers. The system that has been developed here is still largely based on a rigid script. To create a more autonomous robot, this script would have to be replaced with a more dynamical system. An important part of this would be spatial awareness, enabling the robot to locate the audience, the teacher and the beamer. Such a feature could be realised by processing the videostream from the NAO's internal camera or external cameras in the lecture room.


%\nocite{*}
%\bibliographystyle{IEEEtran}
%\bibliography{Bibliography}
%\addcontentsline{toc}{chapter}{Bibliography}

%\appendix
%\input{appendix.tex}

\end{document}